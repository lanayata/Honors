\chapter{Introduction}\label{C:intro}
This project seeks to design, implement, and evaluate an interactive 3D visualisation
software system for displaying the content in the Kepler Exoplanets dataset [28]. 
\\\\
The deliverable is 
\\\\
The resulting system will

\section{Problem Statement}
many planets that have been located outside of our own solar system, these are
called exoplanets
\subsection{Understanding the content in the dataset}
Understanding and analysing large datasets whose size defies simplistic analysis is a known issue that many areas of research are attempting to address from datamining to discover SOMTHING HERE in the data itself to visualisations to convey the information in a visual mannor. All of these research methods involve finding and displaying important aspects of the data so that users can more efficiently use it.
\\\\
The content in the dataset used for this project is made up of records of each exoplanet discovered, each of which contains 46 fields.

\subsection{Comprehension of planatery information}
Much of the information regarding planets is criptic and uninituitive, this make its understandability difficult. Visualisations attempt to address this by displaying the information in a way that conveys the information in a simplisitic way that allows easier user comprehension.
\\\\
something about faster cognition times
\subsection{Effective user interaction with visualisation}
A visualisation that soley displays information without effective methods of interaction will not have the immersive qualities that keeps users engaged as would be the case with an interactive visualisation
\section{Key issues project addresses}
To summarise the above sections, this project addresses the following key issues:
\begin{enumerate}
 \item 
\end{enumerate}

\section{Contributions of this project}
\begin{enumerate}
 \item 
\end{enumerate}
