
\chapter{Project Methodologies}\label{C:m}

\section{Project management approach}
Following a structured project managament approach is important as it avoids the problem caused by following a code-and-fix approach as described  as 1) write some code. 2) Fix the problems in the code \cite{boehm}. By following a process model it encourages thinking about requirements, design, and testing before coding is commenced. 
\\\\
The project methodology chosen for this project was a customised Spiral Model made up of requirements analysis, design, implementation, and evaluation phases as shown in the below figure. The reason for limiting the model to these 4 phases was because....  
\begin{figure}[h!]
  \centering
      \includegraphics[width=0.6\textwidth]{images/spiral_model.png}
  \caption{Spiral process model followed}
\end{figure}
Using a spiral model allowed me to produce a deliverable feature at the end of each iteration of the model, this ensured that I did not become delayed or stuck in my development. By using this methodology it also allowed me to prioritise the features that were the most important to the visualisation which reduced the risk that there would be missing or incomplete components at the end of the project. 

The advantages of this methodolgy over other choices such as the waterfall model or an agile aproach such as Scrum was that it provided me with the benifits of a structured workflow that is a feature of the waterfall model as well as a flexible iterative process that is a feature of Agile methodologies. Following a pure waterfall methodology would not have allowed me to iteratively design, develop, and evaluate each feature which would have forced more upfront design which ........... Following a pure agile approach would not have been optimal as most Agile methodologies are more benificial to projects that have a team working on them. 
\\\\
This project management technique supported the creation of a visualisation as 
\\\\
The choice of this project management apprach meant ....
\\\\
Weekly meetings with the supervisor of the project, Dr Stuart Marshall, were used to provide guidance and 
\\\\
Using using other supporting project management techniques such as Gantt charts [APPENDIX] and work breakdown structures(WBS) [APPENDIX] allowed efficient documentation of planning and work completed in the project as well as displaying the following stages requried to complete the project.

\section{System design approach}
To guide the creation of the visualisation a user oriented design approach was used, in particular making use of user models (personas) were created to create a sense of empathy and understanding for the forseen users of the visualisation in order to better understand the requrirements and design decistions to be made. 

\section{Key difficulties in project}
As this project builds upon a previous system much of the exisiting code and execution flow needs to be modified. This requires understanding of how the system was originally built and designed. Because this system does not have any unit or integration tests, going ahead without a comprehenive knowledge of the core functionality would be foolish.

Encountered errors in Processing framework due to number of elements needing to be displayed on screen. 

Having a time constraint of 300 hours for this project over the course of a year meant that prioritisation of visualsation features needed to be made to ensure that 

