
\chapter{Project Methodologies}\label{C:m}

\section{Project management approach}
% Why Project Managment is important
Project management is the discipline of planning, executing, monitoring, and evaluating a project. It is a vital role in any project as it is the glue between all of the different components that go into a successfull project. It is benificial to projects as it encourages thinking about requirements, design, and testing before coding is commenced. For this project it helped to avoid the problem of following a code-and-fix approach described  as 1) write some code. 2) Fix the problems in the code \cite{boehm}. The code-and-fix approach may be suitable in some very small scale applications, but as soon as a system becomes complex it increases the chance of a system turning into a nightmare of high coupling, low cohesion, no modularity, minimal structure, and little consistancy. 
\\\\
% Dicussion of SDLC

  %Why a SDLC is needed
The project management methodology/software development lifecycle chosen for this project was a customized Spiral Model made up of requirements analysis, design, implementation, and evaluation phases as shown in the figure below. The reason for limiting the model to these 4 phases was because.... ~
\begin{figure}[h!]
  \centering
      \includegraphics[width=0.6\textwidth]{images/spiral_model.png}
  \caption{Spiral process model followed}
\end{figure}
  %Advantages of Spiral
The advantages of this methodology over other choices such as the waterfall model or an agile approach such as Scrum was that it provided me with the benefits of a structured work flow that is a feature of the waterfall model as well as a flexible iterative process that is a feature of Agile methodologies. Following a pure waterfall methodology would not have allowed me to iteratively design, develop, and evaluate each feature as it would have forced more upfront design which limits flexibility and support for changing requirements as was needed for this project. Following a pure agile approach would not have been optimal as most Agile methodologies(ie SCRUM [REFERENCE]~) are more beneficial to projects that have a team working on them. As it was I was agile in my approach to the project as I produced a deliverable at the end of each interation of the spiral model and responded to change in the form of feedback and ideas from the supervisor of the project.
\\\\
  %How Spiral Supported project
This project management technique supported the creation of a visualisation as it allowed the flexibility to add and remove components into the visualisation as they were discovered to be beneficial or not. It also supported the expansion of the project brief to include using a kinect sensor in order to control the visualisation. The choice of this project management approach meant that whilst I had the freedom to explore visualisation options I also had a structured software development life cycle to guide me and provide SOMETHING HERE~ the project through the necessary steps to end in completion of each component.
\\\\
Using a spiral model allowed me to produce a deliverable feature at the end of each iteration of the model which was usualy occured each , thus ensured that I did not become delayed or stuck in my development with nothing to show. By using this methodology it also allowed me to prioritize the features that were the most important to the visualisation which reduced the risk that there would be missing or incomplete components at the end of the project. 
\\\\
% Other tools used to support me
By supporting this project methodology with other project management tools such as Gantt charts [APPENDIX~] and work breakdown structures(WBS) [APPENDIX~], it encouraged efficient documentation of planning and work completed in the project as well as displaying the upcoming stages required to complete the project.
\\\\

%Did I acheive all my goals
Following this project managment approach allowed me to achieve all of the goals that I set, although it did not always happen within the timeframes that I had planned and some deadlines had to be revised. The delays were caused by increasing the scope of the project to include the Microsoft Kinext sensor which required further planning, implementation and testing. Although there were these delays, due to the planning and project managment approach that was used, all project elements were completed.

% Gantt Chart

\section{Design approach}
% Decisions on technology and asthetic

% User Centered Design, user models and scenarios
To aid in the design of the visualisation a user centered design approach was used. This involves personas being created, that represent the envisaged end users of the visualisation. After this scenarios are created which describe a situation that a user persona would be in and the functionality that they would need from the system. This approach is described in more detail in the following chapter

% Designing how it looked

% Meetings with supervisor
Weekly meetings with the supervisor of the project, Dr Stuart Marshall, were used to provide guidance and ideas for innovation of the visualisation throughout the project. These meeting ensured that vital components and deliverables were implemented in the required timeframe and also provided a sounding board for ideas for elements to be included in the visualization. Another important aspect of having an involved supervisor was that he provided me the guidance of an experienced academic which was indispensable when navigating the administrative side of organizing delicate matters such as ethics approval for human evaluation of the visualisation.


\section{Difficulties in project}
As this project builds upon a previous system much of the existing code and execution flow needs to be modified. This requires understanding of how the system was originally built and designed. Because this system does not have any unit or integration tests, going ahead without a comprehensive knowledge of the core functionality would be have led to ineffective planning and errors being introduced into the system.
\\\\
%tools
\\\\
%Making use of tall of the data
\\\\
This project had a time constraint of approximately 300 hours over the course of a year, this meant that I needed effective time management techniques to ensure that I spent the right amount of time on each project element, that I prioritised important tasks, set appropriate deadlines, and planned each stage of the project effectively.
