
\chapter{Project Methodology}\label{C:m}

\section{Project management approach}
% Why Project Management is important
Project management is the discipline of planning, executing, monitoring, and
evaluating a project. It is a vital role as it is the glue between all of the
 components that go into a project. It is beneficial as it
encourages critical thinking about requirements, design, and testing before coding is
commenced. For this project it helped to avoid the problem of following a
code-and-fix approach which is 1) write some code. 2) Fix the problems in
the code \cite{boehm1988spiral}. The code-and-fix approach may be suitable in some 
small scale applications, but as soon as a system becomes complex it increases
the chance of a system turning into a nightmare of high coupling, low cohesion,
no modularity, minimal structure, and little consistency \cite{foote1997big}.
A widely used method of managing IT projects is to use a Software Development
Life Cycle (SDLC), a form of project methodology that details the stages that a project must pass through whilst being completed.

\subsection{Software Development Life Cycle (SDLC)}
  %Why a SDLC is needed
The SDLC chosen for
this project was a customized Spiral Model \cite{boehm1988spiral} made up of
requirements analysis, design, implementation, and evaluation stages as shown in
Figure \ref{fig:spiralModel}. A software project repeatedly iterates through these stages for each functionality or component produced. The first stage of the spiral involves analysing the
requirements of the functionality being created. Following this designs are created and then implemented to produce a deliverable artifact. Finally in the fourth stage the artifact is user tested by one
or more people. This evaluation is in addition to the final user evaluation at the completion fo the project. 

\begin{figure}[H]
  \centering
      \includegraphics[width=0.6\textwidth]{images/spiral_model.png}
  \caption{Spiral process model followed}
  \label{fig:spiralModel}
\end{figure}

This SDLC supported the creation of a visualisation as
it allowed the flexibility to add and remove components into the visualisation
as they were discovered to be beneficial or not. The choice of this project management approach
meant that whilst I had the freedom to explore visualisation options I also had
a structured software development life cycle to guide me and the project through
the necessary steps to end in the successful completion of each
component.
%
Using a spiral model also allowed me to produce a deliverable feature at the end
of each iteration of the model which occurred each week to coincide with a
meeting with the supervisor of the project, thus ensuring that I did not become delayed or stuck
in my development with nothing produced. Using this methodology also
helped to prioritise the features that were the most important to the
visualisation, reducing the risk of missing or incomplete
artifacts at the end of the project. 
Following this SDLC allowed me to achieve all of the goals
that I set, although some time frames and deadlines had to be revised as elements were completed before or behind schedule. 


\section{Other project methodologies explored}
Two other project methodologies were explored before the
Spiral Model was chosen. The alternatives were a loose agile method SCRUM \cite{schwaber2004agile}, and
a strict Structured Method, the Waterfall Model \cite{royce1970managing}. 
%Advantages of Spiral
The advantages of the Spiral model over these alternatives was that it provided
me with the benefits of a structured work flow that is a feature of the
Waterfall Model as
well as a flexible iterative process that is a feature of Agile methodologies
without needing to adhere only to the strict guidelines of either. Following a
pure waterfall methodology would not have allowed me to iteratively
design, develop, and evaluate each feature and would have forced more upfront
design limiting the flexibility and support for changing requirements 
needed for this project. A pure agile approach was not suitable as Agile methodologies are more
beneficial to team based projects. As it was,
I was agile in my approach to the project and embraced changing requirements,
collaborated with my supervisor (the customer) regularly, and emphasized working
software over large amounts of documentation \cite{beck2001agile}. 

\section{Supporting Tools for project}
%Other tools used to support me
Supporting this project methodology with other project management tools such
as Gantt charts [APPENDIX~] and work breakdown structures(WBS) [APPENDIX~]
encouraged efficient documentation of planning and work completed in the project
as well as displaying the upcoming stages required to complete the project.
\\

%Version Control
Version control was important for this project as it mitigated against the risks
of file system crashes and corruption. It also maintained effective revision
history that could be used to backtrack to or view changes that were made
earlier in the project. Version control was also valuable as it allowed me to
maintain multiple branches (versions) of my project that could be swapped between whilst developing multiple features.
This was important for the creation of the Keyboard and mouse system and the
Microsoft Kinect system as they were on separate branches in the repository. 
The version control tool used for this project was Git which allowed
 the repository to be hosted on the repository hosting web service Github. A
limitation of this choice was that the repository was hosted on a free Github license which meant that the
repository would be open to the public to view.

%Meetings with supervisor
Weekly meetings with the supervisor of the project, Dr Stuart Marshall, were
used to provide guidance and ideas for innovation of the visualisation
throughout the project. These meeting ensured that vital components and
deliverables were implemented in the required timeframe and also provided a
sounding board for ideas.
Another important aspect of having an involved supervisor was that he provided the guidance of an experienced academic which was indispensable when
navigating the administrative side of organizing delicate matters such as ethics
approval for the user evaluation of the visualisation.
