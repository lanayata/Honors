
\chapter{Project Methodologies}\label{C:m}

\section{Project management approach}
Following a structured project management approach is important as it avoids the problem caused by following a code-and-fix approach as described  as 1) write some code. 2) Fix the problems in the code \cite{boehm}. By following a process model it encourages thinking about requirements, design, and testing before coding is commenced. 
\\\\
The project methodology chosen for this project was a customized Spiral Model made up of requirements analysis, design, implementation, and evaluation phases as shown in the below figure. The reason for limiting the model to these 4 phases was because....  
\begin{figure}[h!]
  \centering
      \includegraphics[width=0.6\textwidth]{images/spiral_model.png}
  \caption{Spiral process model followed}
\end{figure}
Using a spiral model allowed me to produce a deliverable feature at the end of each iteration of the model, this ensured that I did not become delayed or stuck in my development. By using this methodology it also allowed me to prioritize the features that were the most important to the visualisation which reduced the risk that there would be missing or incomplete components at the end of the project. 
\\\\
The advantages of this methodology over other choices such as the waterfall model or an agile approach such as Scrum was that it provided me with the benefits of a structured work flow that is a feature of the waterfall model as well as a flexible iterative process that is a feature of Agile methodologies. Following a pure waterfall methodology would not have allowed me to iteratively design, develop, and evaluate each feature which would have forced more upfront design which limits flexibility and support for changing requirements and design as was needed for this project. Following a pure agile approach would not have been optimal as most Agile methodologies are more beneficial to projects that have a team working on them. As it was I was agile in my approach to the project as I produced a deliverable at the end of each of the cycles of the spiral model and responded to change in the form of feedback and ideas from the supervisor of the project.
\\\\
This project management technique supported the creation of a visualisation as it allowed the flexibility to add and remove components into the visualisation as they were discovered to be beneficial or not. It also supported the expansion of the project brief to include using a kinect sensor in order to control the visualisation. The choice of this project management approach meant that whilst I had the freedom to explore visualisation options I also had a structured software development life cycle to guide me and provide the project through the necessary steps to end in completion of each component.
\\\\
By supporting this project methodology with other project management tools such as Gantt charts [APPENDIX] and work breakdown structures(WBS) [APPENDIX], it encouraged efficient documentation of planning and work completed in the project as well as displaying the upcoming stages required to complete the project.
\\\\
Weekly meetings with the supervisor of the project, Dr Stuart Marshall, were used to provide guidance and ideas for innovation of the visualisation throughout the project. These meeting ensured that vital components and deliverables were implemented in the required timeframe and also provided a sounding board for ideas for elements to be included in the visualization. Another important aspect of having an involved supervisor was that he provided me the guidance of an experienced academic which was indispensable when navigating the administrative side of organizing delicate matters such as ethics approval for human evaluation of the visualisation.


\section{Key difficulties in project}
As this project builds upon a previous system much of the existing code and execution flow needs to be modified. This requires understanding of how the system was originally built and designed. Because this system does not have any unit or integration tests, going ahead without a comprehensive knowledge of the core functionality would be foolish.
\\\\
Encountered errors in Processing framework due to number of elements needing to be displayed on screen. 
\\\\
Having a time constraint of 300 hours for this project over the course of a year meant that prioritization of visualization features needed to be made to ensure that 
\\\\
Libraries used for gesture detection in kinect are opensource in order to work with processing did not have decent detection