\chapter{Summary}\label{C:con}
This project contributes the design,
implementation, and evaluation of an
interactive 3D visualisation called the Improved Kepler Visualisation Tool
(IKVT) to the field of Human Computer Interaction. IKVT has two modes of interaction; the traditional keyboard \& mouse and
the more novel Microsoft Kinect sensor. 

\section{System Design}
The requirements for IKVT were designed using Cooper et al.'s user oriented
design approach which emphasised using user models and user scenarios
\cite{AboutFace3}.  Two user
models and seven user scenarios were used in the creation of eight project
requirements that
were used extensively in the design, implementation and evaluation of IKVT to
ensure that it
solved the four key issues that this project hoped to address.
\begin{description}
 \item[Issue 1.] Content in the Kepler Exoplanet Database is difficult to view
and
understand due to its amount and labeling.
 \item[Issue 2.] Planetary information is complex and difficult to comprehend
without
a visual reference due to its scale.
 \item[Issue 3.] Existing visualisations for exploring planetary data have
minimal
functionality for exploring the information in effective ways.
 \item[Issue 4.] Visualisations need to allow user interaction to make the most
of
the data they display.
\end{description}

\section{System Implementation}
IKVT was implemented in Processing, an open-source Java programming language and
integrated development environment (IDE). It was built upon an existing
visualisation, the Kepler Visualisation Tool \cite{kepler_github,
kepler_article} to create more effective interaction techniques and allow users
to access more information in the Kepler Exoplanet dataset. During the
implementation stage of the project each of the
designs created in the system design stage were implemented in
fulfillment of the project requirements. 

\section{System Evaluation}
IKVT was evaluated in a user study that gathered both qualitative and
quantitative
results. The evaluation was designed to gather users experiences with the IKVT
to examine whether or not it fulfilled the system requirements designed to solve
the issues mentioned above.

The evaluation found that all of these requirements were fulfilled by the
visualisation. In addition to this the participants in general found that the
IKVT had a low learning curve, was enjoyable to use, and allowed
access to interesting information. There was also a common consensus among the
participants that the Kinect system was more fun to use than the keyboard \&
mouse because of its novelty, but lacked the control that the
keyboard \& mouse allowed which made it less effective at accessing the
information in the visualisation. The evaluation also revealed that one area,
viewing the habitable zones of stars, lacked
usability because it was unintuitive for the users participating in the
evaluation.     
\section{Future Work}
IKVT  can be taken further in many ways depending
on how it is intended to be used. There is the option of using the system as a
terminal that users would use at an observatory or attraction where prior
knowledge of the system is limited and amount of time users would spend on the
system would be small. In this case further expanding the user experience and
improved Kinect interaction would be beneficial as immersion would be the
decider on its success. Another option would be for a standalone
desktop system that users would use multiple times and so prior knowledge of how
to use the system could be expected. This would mean that more complex
functionality could be introduced with the expectation that it could be learned
and used by
users. The systems current state could me modified to fit into either of these
two options.

A weakness in the system that needs to be addressed is the Goldilocks zone view
which the evaluation discovered was not intuitive enough to make it effective.
To
address this the functionality needs to be evaluated to discover which aspects
of it cause it to be unintuitive. The first improvement is to provide a common
point of reference for the habitable zones. This is because the area that
confused users the most was that when they selected a new planet Goldilocks
zones changed. 

The user evaluation found that gesture based control of IKVT was the most
enjoyable way for users to interact with IKVT even though it was less effective
at accessing information. This opens up the opportunity for further work to
improve the level of gesture based control and accuracy so that it rivals the
keyboard \& mouse. This could be done in a range of ways from simply enhancing
the intuitivity, to more advanced methods of gesture detection such as joining
gestures together.
 
These three future improvements are ordered by importance below.
\begin{enumerate}
 \item Address Goldilocks zone view
 \item Improve gesture based control
 \item Focus IKVT towards a set environment to focus futher development
\end{enumerate}

